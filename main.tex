\documentclass[orivec]{llncs}
\usepackage{showkeys}
\usepackage{amsmath}
\usepackage{amssymb}
\usepackage{graphicx}
\usepackage[T1]{fontenc}
\usepackage{listings, framed}
\lstset{
  language=Java,
  showstringspaces=false,
  columns=flexible,
  basicstyle={\ttfamily},
  frame=l,
  numbers=left,
  numberstyle={\ttfamily},
%  breaklines=true,
  breakatwhitespace=true,
  tabsize=3,
  escapechar=|
}
%\usepackage{algorithm, algorithmic}
\usepackage{url}
\usepackage{relsize}
%\usepackage{xcolor}
%\usepackage{multirow}

%\newcommand{\PreserveBackslash}[1]{\let\temp=\\#1\let\\=\temp}
%\newcolumntype{C}[1]{>{\PreserveBackslash\centering}p{#1}}
%\newcolumntype{R}[1]{>{\PreserveBackslash\raggedleft}p{#1}}
%\newcolumntype{L}[1]{>{\PreserveBackslash\raggedright}p{#1}}

%\newif\iflongversion
%\longversionfalse               % for conference version
%\longversiontrue                % for technical report

% \|name| or \mathid{name} denotes identifiers and slots in formulas
\def\|#1|{\mathid{#1}}
\newcommand{\mathid}[1]{\ensuremath{\mathit{#1}}}
% \<name> or \codeid{name} denotes computer code identifiers
\def\codesize{\small}
\def\<#1>{\codeid{#1}}
\newcommand{\codeid}[1]{\ifmmode{\mbox{\codesize\ttfamily{#1}}}\else{\codesize\ttfamily #1}\fi}

\newcommand{\todo}[1]{{\color{red}\bfseries [[#1]]}}

\DeclareMathOperator*{\append}{\bowtie}

\newcommand{\fs}[1]{\todo{FS: #1}}
\newcommand*\xor{\oplus}

% Reduce padding for \boxed
%\setlength{\fboxsep}{.5\fboxsep}

\newtheorem{alg}{Algorithm}

\newcommand{\wrt}{\textit{wrt.\ }}
\newcommand{\ie}{, \textit{ie.}, }

\newcommand{\numberofscoops}{{\#\mathit{scoops}}}
\newcommand{\noncesize}{{\mathit{nonce}_{\mathit{size}}}}
\newcommand{\byte}{{\mathit{byte}}}
\newcommand{\nonce}{{\mathit{nonce}}}
\newcommand{\nonces}{{\mathit{nonces}}}
\newcommand{\scoops}{{\mathit{scoops}}}
\newcommand{\plot}{{\mathit{plot}}}
\newcommand{\seed}{{\mathit{seed}}}
\newcommand{\deadline}{{\mathit{deadline}}}
\newcommand{\generation}{{\mathit{generation}}}
\newcommand{\size}{{\mathit{size}}}
\newcommand{\scoopnumber}{{\mathit{scoopNumber}}}
\newcommand{\data}{{\mathit{data}}}
\newcommand{\challenge}{{\mathit{challenge}}}
\newcommand{\mynext}{{\mathit{next}}}
\newcommand{\nextchallenge}{\challenge_\mynext}
\newcommand{\initialchallenge}{{\mathit{initialChallenge}}}
\newcommand{\height}{{\mathit{height}}}
\newcommand{\trunk}{{\mathit{trunk}}}
\newcommand{\block}{{\mathit{block}}}
\newcommand{\oblivion}{{\mathit{oblivion}}}
\newcommand{\now}{{\mathit{now}}}
\newcommand{\beat}{{\mathit{beat}}}
\newcommand{\weightedbeat}{\mathit{weightedBeat}}
\newcommand{\nextweightedbeat}{\weightedbeat_\mynext}
\newcommand{\power}{{\mathit{power}}}
\newcommand{\previousblockhash}{{\mathit{previousBlockHash}}}
\newcommand{\waitingtime}{{\mathit{waitingTime}}}
\newcommand{\nextwaitingtime}{\tau_\mynext}
\newcommand{\nextpower}{\power_\mynext}
\newcommand{\nextacceleration}{\alpha_\mynext}
\newcommand{\nextblock}{\block_\mynext}
\newcommand{\genesis}{{\mathit{genesis}}}
\newcommand{\consensus}{{\mathit{consensus}}}
\newcommand{\nattobe}{{\mathit{nat2be}}}
\newcommand{\betonat}{{\mathit{be2nat}}}
\newcommand{\finalhash}{{h_{\mathit{final}}}}

\newcommand{\Prologs}{{\mathsf{Prologs}}}
\newcommand{\Plots}{{\mathsf{Plots}}}
\newcommand{\Challenges}{{\mathsf{Challenges}}}
\newcommand{\Scoops}{{\mathsf{Scoops}}}
\newcommand{\Nonces}{{\mathsf{Nonces}}}
\newcommand{\Trunks}{{\mathsf{Trunks}}}
\newcommand{\Blocks}{{\mathsf{Blocks}}}
\newcommand{\GenesisBlocks}{{\mathsf{GenesisBlocks}}}
\newcommand{\NonGenesisBlocks}{{\mathsf{NonGenesisBlocks}}}
\newcommand{\Deadlines}{{\mathsf{Deadlines}}}

%
\begin{document}
%
\begin{frontmatter}
  \title{A Formalization of Signum's Consensus}
\author{Fausto Spoto}
\institute{Dipartimento di Informatica, Universit\`a di Verona, Verona, Italy\\
  \email{fausto.spoto@univr.it}}
%
\maketitle
%
\begin{abstract}
  Very interesting text.
\end{abstract}
%
\end{frontmatter}

\section{Introduction}\label{sec:introduction}

A blockchain is a list of \emph{blocks}, each reporting the hash
of its previous block, satisfying some consistency or \emph{consensus} rules.
Blocks hold \emph{transactions}, whose exact nature is not relevant here.
In general, they are requests to update the state of a global abstract machine:
a ledger of payments (as in Bitcoin~\cite{Nakamoto08,Antonopoulos17})
or a sort of global RAM where data structures are allocated and modified
(as in Ethereum~\cite{AntonopoulosW18}).
By using hashes as machine-independent
pointers, blockchains can be distributed in a network of peers.
This is desirable since data gets safely duplicated
and no special peer determines the history alone.
However, peers expand the blockchain at will, independently from the other
peers, hence the blockchain becomes a tree rather than a list.
A notion of chain quality is used to incentivize peers to append blocks to the highest-quality chain
(the \emph{best} chain).
Therefore, a peer might replace its current best chain with another, even better chain,
a so called \emph{history change}.

As presented above, peers are free to generate new blocks at maximal speed, flood the network
with new blocks and make difficult the emergence of a best chain, with frequent history changes.
This is an efficiency and security issue: history changes allow
\emph{double spending}, when the same money is moved in the ledger twice, once in the current history
and once in the updated history. The actual genious of Nakamoto~\cite{Nakamoto08} was to
(largely) solve this issue with a consensus rule requiring blocks to answer a \emph{challenge}
contained in their previous block. Namely, the binary hash of each block must be smaller than
a \emph{difficulty} value computed at the previous block, directly bound to the quality of the chain.
Therefore, who creates (\emph{mines}) of a new block runs a \emph{proof of work} algorithm
that rotates (\emph{grinds}) many alternative values for a block field, called \emph{nonce}, until
the hash of the block is smaller than the difficulty. This complicates the creation of new blocks,
makes it impossible to create blocks at arbitrary speed and introduces an
incentive to expanding the best chain only, rather than creating alternative histories by mining at multiple chains,
since otherwise the miner
risks spending work (concretely, electricity) for creating blocks discarded by its peers.
A miner gets remunerated for its work whenever it finds a new block before all other miners.

The \emph{proof of work} is a brute-force algorithm,
because of the non-correlation property of hash functions.
Therefore, it consumes energy: Bitcoin is said to consume as much electricity as
a medium-sized country; moreover, it is not egalitarian, being
worthwhile only in countries where electricity is cheap; furthermore, it
runs more efficiently over dedicated, relatively expensive hardware (such as ASICs),
against the promise of a democratic and open network.
Therefore, the current trend is towards the \emph{proof of stake}.
This comes in different flavors, but
the common idea is that mining is limited to a (static or dynamic, exclusive or delegatable)
set of peers, that \emph{stake} a collateral to buy mining rights.
Many criticize proof of stake for being centralized and undemocratic
(\emph{rich becomes richer}).
Moreover, it suffers from what we call a \emph{start-up issue}: as long as the cryptocurrency
of a newborn blockchain has still no value, it is hard to convince miners to work and
be updated, since there is no incentive for that. Moreover, miners in
a proof of stake blockchain get punished (\emph{slashed}) if they misbehave or are offline, which
might be perceived as unfair if that happens because of a connectivity issue or black-out.

An alternative to proof of work and proof of stake is
\emph{proof of space}~\cite{AtenieseBFG14,DziembowskiFKP15}, where
miners must dedicate a large chunk of disk memory for answering challenges.
Its energy consumption is close to zero and no special
hardware helps for mining, currently: the technology is both cheap
and democratic. Moreover, proof of space allows
one to capitalize on unused memory, for free, while proof of work has always an
inherent electricity cost.
For fairness, proof of space protocols should only allow to generate answers of
quality directly proportional to the allocated space, or otherwise they are said
to suffer from a space/memory tradeoff.
Moreover, \cite{ParkKFGAP18} shows that
cheap answers actually introduce new security attacks,
known as \emph{nothing-at-stake} problems, such as \emph{block grinding},
\emph{challenge grinding} and \emph{mining on multiple chains},
and proposes solutions to cope with them.
These attacks increase the risk of double spending and make it convenient
(\emph{rational}) to mine through space \emph{and} work, therefore neutralizing the benefits
of proof of space.
Another nothing-at-stake problem, that has not received great attention up to now,
is the \emph{newborn attack}~\cite{TangZDWLG0L19}: a miner that has allocated a large space
for mining for a blockchain network $N$ could use the same space, unchanged, for
mining for a newborn, small network $N'$. If the total space used by the peers of $N'$ is
initially relatively small, it could be possible for the miner to hijack the history of $N'$,
effectively taking its control. Note that these attacks are anti-economic with proof of work,
since computing power can only be dedicated to a mining task.

Most theoretical formalizations of proof of space are
based on challenges against graphs of high pebbling complexity, but
no actual blockchain has ever been built using such theory: only SpaceMint, a prototype and non-maintained
implementation of the protocol, exists~\cite{ParkKFGAP18}. This may be because of the
complex protocol, with an initialization phase
that writes a quite large proof (megabytes) in blockchain,
for each new miner that joins the network.
The only full-fledged implementation of a proof of space blockchain is Signum~\cite{Signum}, that is
not based on graph pebbling and has no such protocol issue, but it has never been formalized up to now.
The authors of~\cite{ParkKFGAP18} warned the developers of Signum about a potential block grinding attack and
a time/memory tradeoff, that the developers have allegedly solved,
but nothing has been proved nor published about it.

This paper provides the following contributions to Signum's algorithm:
%
\begin{itemize}
\item a formalization of the algorithm, recostructed and interpolated from its very informal
  web description~\cite{SignumPlotting} and its poorly commented source code~\cite{SignumSource};
\item a proof that the block grinding attack hinted in~\cite{ParkKFGAP18} cannot actually occur;
\item a protection against challenge grinding and mining on multiple chains attacks, inspired by~\cite{ParkKFGAP18};
\item a protection against newborn attacks.
\end{itemize}
%
These results are relevant since they show that the only implementation
of a blockchain, purely based on proof of space, is actually supported by a formal theory
and can be protected from a large class of attacks.

The rest of this paper is organized as follows.
Sec.~\ref{sec:related_work} discusses related work.

\mbox{}\\

\textbf{Acknowledgments:}
We thank the developers of Signum for discussions at their Discord channel (where, however, most
of our questions remained unanswered) and K.\ Pietrzak for discussing solutions to
grinding attacks and theoretical issues about Signum.

\section{Nonces and Plots}\label{sec:nonces_and_plots}

This section formalizes the notion of nonce and plot, together with their
algorithmic construction. Tab.~\ref{tab:notations} collects some notations,
introduced here and used throughout the paper.
It also reports the specific choices
made in~\cite{SignumPlotting} for such notations, but this paper remains parametric
\wrt them.

\begin{table}[t]
\begin{center}
\begin{tabular}{|c|c|c|}
  \hline
  \textbf{notation} & \textbf{meaning} & \textbf{in~\cite{SignumPlotting}}\\\hline\hline

  $\append$ & concatenation on sequences & \\\hline
  $\numberofscoops$ & number of scoops contained in a nonce & $4096$\\\hline

  $h_\deadline$ & hashing function for computing nonces, plots and deadlines & shabal256\\\hline

  $h_\generation$ & hashing function for computing the generations of challenges & shabal256\\\hline

  $\kappa$ & maximal number of bytes fed to $h_\deadline$ & $4096$\\\hline

\end{tabular}
\end{center}
\caption{Notations used in our formalization and their specific instantiations
  used in~\cite{SignumPlotting}.}
\label{tab:notations}
\end{table}
%
\begin{definition}[Concatenation operator $\bowtie$]
  Sequences are concatenated by $\bowtie$. The same $\bowtie$
  is used to concated a sequence to a element or an element to a sequence.
\end{definition}
%
\begin{definition}[$\nattobe$ and $\betonat$]
  The operators $\nattobe$ and $\betonat$ transform natural numbers
  into their big-endian representation, and vice versa.
\end{definition}
%
\begin{definition}[Hashing function]
  A \emph{hashing function} $h$ of $\size>0$
  is a total map $h:\mathit{byte}^*\to\mathit{byte}^\size$, where
  $\mathit{byte}^\size$ is a sequence of $\size$ bytes, called a \emph{hash} for $h$.
  If $h$ is a hashing function, then $\size(h)$ is its size.
\end{definition}
%
The definition of nonces and plots is parametric \wrt a hashing
function $h_\deadline$ used for their creation.
A \emph{scoop} is a pair of hashes.
A \emph{nonce} is a non-negative \emph{progressive number} $p$, and
a list of $\numberofscoops>0$ scoops or, equivalenty,
a list of $2\times\numberofscoops$ hashes for $h_\deadline$.
%
\begin{definition}[Scoop, Nonce]
  The sets of \emph{scoops} and \emph{nonces} are
  \[
  \Scoops=\left\{\langle h_1,h_2\rangle\left|\begin{array}{l}
  h_1,h_2\text{ are hashes for }h_\deadline
  \end{array}\right.\right\},
  \]
  \[
  \Nonces=\left\{\langle p,\scoops\rangle\left|\begin{array}{l}
  p\in\mathbb{N}\text{ and }\scoops\in\Scoops^\numberofscoops
  \end{array}\right.\right\}.
  \]
\end{definition}
%
A \emph{prolog} is the identifier of the creator of nonces and plots.
For now, it is just a sequence of bytes. Later, we will give structure to prologs
and see how they can be useful.
%
\begin{definition}[Prolog]\label{def:prolog}
  The set of \emph{prologs} is
  \[
  \Prologs=\{\pi\mid\pi\in\mathit{bytes}^*\}.
  \]
\end{definition}
%
The following algorithm constructs a nonce, given its progressive number and a prolog.
It uses a constant $\kappa>0$ to limit its computational cost, hence avoiding to hash
very large chunks of data.
%
\begin{alg}[$\nonce(p,\pi)$]\label{alg:nonce_construction}
  Given $p\in\mathbb{N}$ and $\pi\in\Prologs$, let
  \[
  \nonce(p,\pi)=\langle p,\langle h_0,h_1\rangle\append
  \cdots\append\langle h_{2\times\numberofscoops-2},h_{2\times\numberofscoops-1}\rangle\rangle\in\Nonces,
  \]
  where the hashes $h_0,\ldots,h_{2\times\numberofscoops-1}$ are constructed as follows.
  %
  \begin{enumerate}
  \item Let $\seed=\pi\append\nattobe(p)$.
  \item For each $i$ from $2\times\numberofscoops-1$ to $0$,
    let\footnote{In~\cite{SignumPlotting}, it is said to
    take the \emph{last} $\kappa$ bytes, but an inspection of their code
    shows that they actually take the \emph{first} $\kappa$ bytes. They
    probably use \emph{last} here in the sense of \emph{more recently computed}.}
    \[
    h_i=h_\deadline\left(\text{first $\kappa$ bytes of}\left(\left(\append_{i<j<2\times\numberofscoops}h_j\right)\append\seed\right)\right).
    \]
  \item Let
    \[
    \finalhash=h_\deadline\left(\left(\append_{0\le j<2\times\numberofscoops}h_j\right)\append\seed\right).
    \]
  \item For each $i$ from $0$ to $2\times\numberofscoops-1$, reassign
    $h_i$ to $h_i\xor\finalhash$.
  \item For each odd $i$ from $1$ to $2\times\numberofscoops-1$, swap
    $h_i$ with $h_{2\times\numberofscoops-i}$.
  \end{enumerate}
\end{alg}
%
A \emph{plot} is a set of nonces constructed with Alg.~\ref{alg:nonce_construction},
for a finite non-empty set of progressive numbers $P$ and
for a given prolog $\pi$, recorded in the plot.
%
\begin{definition}[Plot]\label{def:plot}
  The set of \emph{plots} is defined as
  \[
  \Plots=\left\{\left<\pi,\nonces\right>\left|\begin{array}{l}
  \pi\in\Prologs,\ \varnothing\not=P\subset\mathbb{N}\text{ is finite}\\
  \text{and }\nonces=\left\{\nonce(p,\pi)\left|\;p\in P\right.\right\}
  \end{array}\right.\right\}.
  \]
\end{definition}

\section{Challenges and Deadlines}\label{sec:challenges_and_deadlines}

A \emph{challenge} specifies a puzzle that must be solved in order to mine
a new block. In Signum, challenges become a query that can be asked to a nonce, resulting
in an answer called \emph{deadline}.
%Challenges will be applied to plots by taking the \emph{smallest} deadline over their nonces.
%
\begin{definition}[Challenge]\label{def:challenge}
  The set of \emph{challenges} is
  \[
  \Challenges=\left\{\langle\scoopnumber,\sigma\rangle\left|
  \begin{array}{l}
    0\le\scoopnumber<\numberofscoops\\
    \text{and $\sigma$ is a hash for $h_\generation$}
  \end{array}
  \right.\right\}.
  \]
  The $\sigma$ component of a challenge is said to be its \emph{generation signature}.
  In the following, generation signature will be used as a synonym of hash for $h_\generation$.
\end{definition}
%
Given a challenge and a nonce, the latter has a value that specifies how well
the nonce answers the challenge.
%
\begin{definition}[$\mathit{value}(\nonce,\challenge)$]\label{def:nonce_value}
  Let $\nonce\in\Nonces$ and $\challenge\in\Challenges$.
  The \emph{value of $\nonce$ \wrt $\challenge$} is defined as
  %
  \begin{multline*}
    \mathit{value}(\nonce,\challenge)\\
    =h_\deadline(\nonce.\scoops[\challenge.\scoopnumber]\append\challenge.\sigma).
  \end{multline*}
\end{definition}
%
The answer to a challenge could actually be a nonce $n$, whose quality is its value.
But nonces are relatively big (around $262$ kbytes under the assumptions in the
rightmost column of Tab.~\ref{tab:notations}). Since answers are stored in blockchain,
\cite{SignumPlotting}~introduces \emph{deadlines}, a much smaller representation of the value of
$n$, carrying the information needed to reconstruct $n$ and verify that it actually
answers the challenge.
%
\begin{definition}[Deadline]\label{def:deadline}
  The set of \emph{deadlines} is
  \[
  \Deadlines=\left\{
  \langle p,\pi,\mathit{value},\challenge\rangle
  \left|\begin{array}{l}
  \pi\in\Prologs,\ p\in\mathbb{N},\\
  \mathit{value}\text{ is a hash for }h_\deadline\\
  \text{and }\challenge\in\Challenges
  \end{array}
  \right.
  \right\}.
  \]
  Deadlines are totally ordered by increasing value.
\end{definition}
%
Intuitively, the value of a deadline expresses how many milliseconds
must be waited until the deadline expires and a new block can be mined.
However, if the mining power of the network increases, the minimal value of the deadlines
generated by the network tends to decrease and the block creation rate would not be
fixed to $\beat$ (Tab.~\ref{tab:notations}), on average.
This explains why the deadlines value is modulated \wrt an
\emph{acceleration}\footnote{In~\cite{SignumPlotting} the term
\emph{base target} is used for it, but we think that \emph{acceleration} is clearer.},
which is the inverse of Bitcoin's difficulty.
%
\begin{definition}[Deadline's waiting time]\label{def:deadline_waiting_time}
  Given $\delta\in\Deadlines$ and an \emph{acceleration}
  $\alpha\in\mathbb{N}$ such that $\alpha>0$, the
  \emph{waiting time} for $\delta$ \wrt $\alpha$ is\footnote{
  In~\cite{SignumPlotting}, the divisor is actually
  $2^{\size(h_\deadline)-8}\cdot\alpha$, to avoid using very large values for
  $\alpha$. This is theoretically irrelevant and we prefer our simpler presentation.
  }
  %
  \[
  \waitingtime(\delta,\alpha)
  %  =\betonat\left(\text{first $8$ bytes of}\left(\nattobe\left(\frac{\betonat(\delta.\mathit{value})}{\alpha}\right)\right)\right).
  =\frac{\betonat(\delta.\mathit{value})}{\alpha}.
  \]
\end{definition}
%
Def.~\ref{def:deadline_from_nonce} finally shows how a nonce answers a challenge with a deadline.
%
\begin{definition}[$\delta(\nonce,\pi,\challenge)$]\label{def:deadline_from_nonce}
  Given $\nonce\in\Nonces$, $\pi\in\Prologs$ and $\challenge\in\Challenges$, the
  \emph{deadline computed from $\nonce$ for $\pi$ and $\challenge$} is
  $\delta(\nonce,\pi,\challenge)=\langle\nonce.p,\pi,\mathit{value}(\nonce,\challenge),\challenge\rangle$.
\end{definition}
%
Def.~\ref{def:deadline_from_nonce} extends to plots. Remember that plots are non-empty
(Def.~\ref{def:plot}) and embed the identifier of their creator $\pi$;
and that deadlines are ordered by their value.
%
\begin{definition}[$\deadline(\plot,\challenge)$]\label{def:deadline_from_plot}
  Given $\plot\in\Plots$ and $\challenge\in\Challenges$, the \emph{deadline computed
  from $\plot$ for $\challenge$} is\footnote{If more nonces of the plot lead to deadlines
  with the same value, we assume that Def.~\ref{def:deadline_from_plot} chooses one,
  according to some policy that is irrelevant here.}
  \[
  \delta(\plot,\challenge)=\min\limits_{\nonce\in\plot.\nonces}\delta(\nonce,\plot.\pi,\challenge).
  \]
\end{definition}
%
A deadline is valid when the nonce built from its progressive and prolog
has the same value as the deadline \wrt its challenge.
%
\begin{definition}[Deadline's validity]\label{def:deadline_validity}
  Given $\delta\in\Deadlines$, it is \emph{valid} if and only if
  $\delta.\mathit{value}=\mathit{value}(\nonce(\delta.p,\delta.\pi),\delta.\challenge)$.
\end{definition}
%
%Later, the consensus rules of the blockchain (Def.~\ref{def:blockchain})
%will check that deadlines are valid. Therefore, one could be tempted to
%avoid such check by removing the $\mathit{value}$ field from deadlines
%(Def.~\ref{def:deadline}) and computing
%their value, on demand, by Def.~\ref{def:deadline_validity}. However, the
%computation of
%$\nonce(\delta.p,\delta.\pi)$ is relatively expensive in terms of amount of hashed data
%(Def.~\ref{alg:nonce_construction}).
%Therefore, the original informal description~\cite{SignumPlotting} puts the value in the deadlines
%and we follow that approach.

\section{Blockchain Construction}\label{sec:blockchain_construction}

The blocks of the blockchain contain information used for consensus,
called \emph{trunk} by borrowing this terminology from~\cite{CohenP19},
other information such as the previous block hash,
and extra information that is irrelevant here, such as a list of transactions.
Transactions are not formalized below, since they are not used by Signum's consensus.
Blocks can be genesis and non-genesis. Both contain their time of creation and their acceleration.
Genesis blocks have nor trunk nor parent; their height is implicitly $0$.
Challenges $c$ are generated in sequence: there is an initial constant challenge for the genesis
blocks, while the subsequent challenges are generated from the trunk of
each non-genesis block $b$ of the blockchain.
In particular, $c$ is \emph{not} computed from the transactions in $b$,
in order to avoid block-grinding attacks (Sec.~\ref{sec:related_work}).
A deadline that answers $c$ is recorded in the trunk of the sons of $b$.
%
\begin{definition}[Trunk, Block]\label{def:trunk}
  The sets of \emph{trunks} and \emph{blocks} are
  \[
  \Trunks=\left\{\langle\height,\delta\rangle\left|\;\height\in\mathbb{N}\text{ and }\delta\in\Deadlines\right.\right\}
  \]
  \[
  \GenesisBlocks=\left\{\langle\tau,\alpha\rangle\mid\tau\in\mathbb{N},\ \alpha\in\mathbb{N}\text{ and }\alpha>0\right\}
  \]
  \[
  \NonGenesisBlocks=\left\{\left\langle\begin{array}{c}
  \tau,\alpha,\\
  \power,\weightedbeat,\\
  \trunk,\\
  \previousblockhash
  \end{array}\right\rangle\left|\begin{array}{l}
  \tau\in\mathbb{N},\\
  \alpha\in\mathbb{N},\ \alpha>0,\\
  \power\in\mathbb{N},\\
  \weightedbeat\in\mathbb{N},\\
  \trunk\in\Trunks,\\
  \previousblockhash\text{ is a}\\
  \qquad\text{hash for $h_\block$}
  \end{array}\right.\right\}
  \]
  \[
  \Blocks=\GenesisBlocks\cup\NonGenesisBlocks.
  \]
\end{definition}
%
If $b$ is a block, then $b.\tau$ is its creation time (milliseconds from the Unix epoch)
and $b.\alpha$ is the acceleration at $b$. If $b$ is a non-genesis block, then
$b.\power$ expresses how much space has been used to build the path that leads to $b$,
starting from the genesis block; it will be used to select the \emph{best chain} for mining
$b$'s son. The value of $b.\weightedbeat$ is the average of the interval creation time
for the path that leads to $b$; it weighs the last blocks more. It will be
compared to $\beat$ (Tab.~\ref{tab:notations}) to understand if the acceleration
must be increased or decreased in $b$'s son.
The value of $b.\previousblockhash$ is the hash of the previous block in the path leading to $b$.
If $b$ is a genesis block, we abuse notation and assume that $b.\power=b.\weightedbeat=0$.

\begin{definition}[Block's height]\label{def:block_height}
  Let $b\in\Blocks$. The \emph{height} of $b$ is defined as
  \[
  \height(b)=\begin{cases}
  0 & \text{if $b\in\GenesisBlocks$}\\
  b.\trunk.\height & \text{if $b\in\NonGenesisBlocks$.}
  \end{cases}
  \]
\end{definition}
%
Def.~\ref{def:initial_challenge} shows how the first challenge is defined, for genesis blocks.
It is a constant that only depends on contextual values (Table~\ref{tab:notations}).
%
\begin{definition}[$\initialchallenge$]\label{def:initial_challenge}
  The \emph{initial challenge} is the constant\footnote{In~\cite{SignumPlotting},
  the scoop number is derived from $\sigma_\genesis$; we simplify it to $0$, without loss
  of generality.}
  %
  \[
    \initialchallenge=
    \langle 0,\sigma_\genesis\rangle
    %\\\langle\betonat(h_\generation(\sigma_\genesis\append\nattobe(1)))\text{ mod }\numberofscoops,\sigma_\genesis\rangle,
  \]
  %
  where $\sigma_\genesis$ is a constant generation signature used for the genesis of the blockchain
  (see Table~\ref{tab:notations}).
\end{definition}
%
Def.~\ref{def:next_challenge_from_trunk}
shows how a challenge is derived from the trunk of a non-genesis block.
%
\begin{definition}[$\nextchallenge(\trunk)$]\label{def:next_challenge_from_trunk}
  Let $\trunk\in\Trunks$. The \emph{next challenge for $\trunk$} is
  \begin{multline*}
    \nextchallenge(\trunk)=\\
    \langle\betonat(h_\generation(\sigma\append\nattobe(\trunk.\height+1)))\text{ mod }\numberofscoops,\sigma\rangle
  \end{multline*}
  where
  \[
  \sigma=h_\generation(\trunk.\delta.\challenge.\sigma\append\trunk.\delta.\pi).  
  \]
\end{definition}
%
The definition of the generation signature $\sigma$ for the next challenge,
in Def.~\ref{def:next_challenge_from_trunk},
has puzzled us for some time, since~\cite{SignumPlotting} appends a \emph{previous block generator}
to the previous block's generation signature $\trunk.\delta.\challenge.\sigma$.
That concept, however, is defined nowhere.
We had to dive in the source code of the Signum node
to understand that it is actually
an identifier (more concretely, the public key)
of the creator of the deadline for the previous block
(see \url{https://github.com/signum-network/signum-node/blob/main/src/brs/GeneratorImpl.java}, constructor of \<GeneratorStateImpl>).
We abstract that information through the prolog of the previous deadline, hence this is why
Def.~\ref{def:next_challenge_from_trunk} appends
$\trunk.\delta.\pi$ to define $\sigma$.

Later, it will be handy to determine the next challenge for a block. Note that
the next definition only uses the trunk inside the block.
%
\begin{definition}\label{def:next_challenge_from_block}
  Let $b\in\Blocks$. Its \emph{next challenge} is
  \[
  \nextchallenge(b)=\begin{cases}
  \initialchallenge & \text{if $b\in\GenesisBlocks$}\\
  \nextchallenge(b.\trunk) & \text{if $b\in\NonGenesisBlocks$.}
  \end{cases}
  \]
\end{definition}

In the following, it is shown how the information inside a block is used to construct
the information inside its son(s). The computation of the next weighted beat
gives more or less weight to the previous weighted beat, depending on a constant $\oblivion$.
Therefore, the latter ($0\le\oblivion\le 1$) expresses how quickly the acceleration reacts to changes
in mining power. The computation of the next power uses the same formula of Bitcoin~\cite{WalkerG24},
adapted to our context: the maximal (hence worse) deadline's value $2^{8\cdot\size(h_\deadline)}$ is compared
to the actual deadline's value and their ratio expresses how much space has been used to compute the deadline.
%
\begin{definition}[Next functions]\label{def:next}
  Let $b\in\Blocks$ and $\delta\in\Deadlines$. We define
  \[
  \nextwaitingtime(b,\delta)=b.\tau+\waitingtime(\delta,b.\alpha)
  \]
  \[
  \nextweightedbeat(b,\delta)=\begin{array}{c}
  \waitingtime(\delta,b.\alpha)\cdot\oblivion\\
  +b.\weightedbeat\cdot(1-\oblivion)
  \end{array}
  \]
  \[
  \nextacceleration(b,\delta)=\frac{b.\alpha\cdot\nextweightedbeat(b,\delta)}{\beat}
  \]
  \[
  \nextpower(b,\delta)=b.\power+\frac{2^{8\cdot\size(h_\deadline)}}{\betonat(\delta.\mathit{value})+1}.
  \]
\end{definition}

We are finally ready to show how the next block of $b$ is constructed, once
a deadline $\delta$ has been chosen for it.
%
\begin{definition}[Next block]\label{def:next_block}
  Let $b\in\Blocks$ and $\delta\in\Deadlines$. We define
  $\nextblock(b,\delta)\in\NonGenesisBlocks$ as
  \[
  \nextblock(b,\delta)=\left\langle\begin{array}{c}
  \nextwaitingtime(b,\delta),\nextacceleration(b,\delta),\\
  \nextpower(b,\delta),\nextweightedbeat(b,\delta),\\
  \langle\height(b)+1,\delta\rangle,\\
  h_\block(b)
  \end{array}\right\rangle,
  \]
  where $h_\block(b)$ is the application of $h_\block$ to the byte representation of $b$.
\end{definition}

A blockchain is a set of blocks, linked through their $\previousblockhash$ field.
It must contain exactly one genesis block; it has no hash collisions among its blocks;
and all its blocks must satisfy the \emph{consensus rules}.
%
\begin{definition}[Blockchain, Consensus]\label{def:blockchain}
  A \emph{blockchain} is a set $B\subset\Blocks$ such that:
  \begin{enumerate}
  \item\label{prop:blockchain:genesis} there is exactly one $b\in B\cap\GenesisBlocks$, written as $\genesis(B)$;
  \item\label{prop:blockchain:no_collision} for each hash $h$ of $h_\block$, there is at most
    one $b\in B$ such that $h_\block(b)=h$, written as $\block(B,h)$;
  \item\label{prop:blockchain:consensus} for each $b\in B$, the predicate $\consensus(B,b)$ holds, where
    %
    \begin{itemize}
    \item if $b\in\GenesisBlocks$, then the only requirement for consensus is that
      $b$ is not created in the future:
      \[
      \consensus(B,b)=b.\tau\le\tau_\now;
      \]
    \item if $b\in\NonGenesisBlocks$, then $\consensus(B,b)$ is the logical conjunction
      of all the following consensus rules:
      \begin{enumerate}[label=(\alph*)]
      \item\label{prop:consensus:no_future} $b$ is not created in the future:
        \[
        b.\tau\le\tau_\now;
        \]
      \item\label{prop:consensus:valid} the deadline of $b$ (that is, $b.\trunk.\delta$) is valid (Def.~\ref{def:deadline_validity});
      \item\label{prop:consensus:no_dangling} there are no dangling pointers:
        \[
        p=\block(B,b.\previousblockhash)\text{ exists;}
        \]
      \item\label{prop:consensus:answer} the deadline of $b$ answers the challenge of $p$ (Def.~\ref{def:next_challenge_from_block}):
        \[
        \nextchallenge(p)=b.\trunk.\delta.\challenge;
        \]
      \item\label{prop:consensus:next_block} $b$ is the next block of $p$ \wrt the deadline of $b$ (Def.~\ref{def:next_block}):
        \[
        b=\nextblock(p,b.\trunk.\delta).
        \]
      \end{enumerate}
    \end{itemize}
    %
  \end{enumerate}
\end{definition}
%
The above consensus rules, reconstructed and interpolated
from~\cite{SignumPlotting,SignumSource},
do not constrain the prolog of the deadlines in any way:
each block can have an arbitrary prolog. Later, it will be shown why it is useful to
restrain prologs with an extra consensus rule.
%
\begin{definition}[Blockchain network]\label{def:blockchain_network}
  A \emph{blockchain network} is a
  network of \emph{peers} (computers), each connected to the other peers,
  each holding its own version of a blockchain, for the same genesis block.
  Each peer holds a plot (Def.~\ref{def:plot}) in its memory, starts with
  a blockchain that only holds a single block and runs two
  algorithms, concurrently: the \emph{block mining} algorithm
  and the \emph{block mined} algorithm.
\end{definition}
%
Def.~\ref{def:blockchain_network} simplifies the picture very much:
peers are fully connected, never disconnect and never need to synchronize.
Moreover, in practice, peers do not hold plots but rather rely on (one or more)
external services (miners) that hold one or more plots.
The goal here is to keep the picture as simple as possible and concentrate on the properties of
Signum's consensus only: Def.~\ref{def:blockchain_network} does not pretend
to describe a real blockchain implementation.

The block mining algorithm looks for the most powerful block in blockchain,
mines a new block on top of it, adds it to the blockchain and whispers it to the connected peers.
%
\begin{alg}[Block mining]\label{alg:block_mining}
  The \emph{block mining} algorithm of a peer $P$, holding blockchain $B$,
  is the following infinite loop:
  %
  \begin{enumerate}
  \item\label{step:block_mining:loop} identify a most powerful\footnote{In theory, more blocks might be the most powerful in blockchain, although this is highly unlikely; in that case, any of them will be chosen.} block $p$ in $B$;
  \item\label{step:block_mining:next_challenge} compute $c=\nextchallenge(p)$ (Def.~\ref{def:next_challenge_from_block});
  \item\label{step:block_mining:next_deadline} compute $\delta'=\delta(\plot,c)$ (Def.~\ref{def:deadline_from_plot}), where $\plot$
    if the plot of $P$;
  \item\label{step:block_mining:next_block} compute $b'=\nextblock(p,\delta')$ (Def.~\ref{def:next_block});
  \item\label{step:block_mining:wait} wait until $b'.\tau\le\tau_\now$;
  \item\label{step:block_mining:add} add $b'$ to $B$;
  \item whisper $b$ to the peers connected to $P$;
  \item go back to step~\ref{step:block_mining:loop}.
  \end{enumerate}
\end{alg}
%
The block mined algorithm receives a block whispered from a connected peer, checks its validity
and adds it to the blockchain.
%
\begin{alg}[Block mined]\label{alg:block_mined}
  The \emph{block mined} algorithm of a peer $P$, holding blockchain $B$,
  is the following infinite loop:
  %
  \begin{enumerate}
  \item\label{step:block_mined:loop} wait for a block $b$ whispered from a connected peer $P'$;
  \item\label{step:block_mined:add} if $B\cup\{b\}$ is a blockchain, add $b$ to $B$;
  \item go back to step~\ref{step:block_mined:loop}.
  \end{enumerate}
\end{alg}
%
In practice, step~\ref{step:block_mined:add} of Alg.~\ref{alg:block_mined} should
allow the addition only of blocks $b$ that are \emph{powerful enough} to look useful,
in order to avoid filling the memory with useless blocks. This is not relevant in this paper.
Moreover, if the whispered block $b$
at step~\ref{step:block_mined:add} of Alg.~\ref{alg:block_mined} is more
powerful than $b'$ at step~\ref{step:block_mining:wait} of Alg.~\ref{alg:block_mining},
a rational peer would interrupt the wait at step~\ref{step:block_mining:wait}
of Alg.~\ref{alg:block_mining}, discard $b'$ and restart Alg.~\ref{alg:block_mining}
from step~1, since the whispered $b'$ is better than the block $b$ that it is trying to mine,
hence it is wiser to stop wasting time and start mining on top of the new best block $b'$.
Such optimizations are not considered here.

Note that peers check the validity of blocks coming from outside
(step~\ref{step:block_mined:add} of Alg.~\ref{alg:block_mined}), since they do not trust
their connected peers. Instead, they do not check the validity of the blocks that they
mine themselves (step~\ref{step:block_mining:add} of Alg.~\ref{alg:block_mining}),
since they are valid by construction, as shown below. This
guarantees that $B$ remains a blockchain in every peer, as Prop.~\ref{prop:mining_is_sound} shows.
The hypothesis on no hash collisions is standard for blockchains.
%
\begin{proposition}\label{prop:mining_is_sound}
  If no hash collision occurs at step~\ref{step:block_mining:add} of Alg.~\ref{alg:block_mining},
  then the set $B$ of blocks in each peer is a blockchain.
\end{proposition}

\section{Attacks and Protections}\label{sec:attacks_and_protections}

\section{Related Work}\label{sec:related_work}

The proof of work has been developed, originally, for combatting email spam~\cite{DworkN92}:
email senders must compute some
work to have emails accepted by their recipient. The input includes
the address of the recipient and the date of sending, in order to ban
recycling of work. The algorithm adds extra data at the end of the email,
which corresponds to the nonce used in Bitcoin.
Also Ethereum started as a proof of work blockchain~\cite{AntonopoulosW18} but has later
moved to a form of proof of stake. The latter can be seen as a
Byzantine consensus algorithm, as pioneered by Tendermint~\cite{Kwon14}.
Most current blockchains use some form of proof of stake nowadays.

The theoretical background of proof of space has been developed, independently,
in the two seminal papers~\cite{AtenieseBFG14} and~\cite{DziembowskiFKP15}.
They feature similarities but also significant differences. Both are based
on directed acyclic graphs (DAGs) of high pebbling complexity.
Pebbling, here, is a directed decoration, with hashes, of the nodes of the DAG, as done for
a Merkle tree.
A prover must keep such a (big) DAG and its pebbling on disk, in order to answer, efficiently,
\emph{challenges} proposed by a verifier, with compact proofs that should convince the latter that
the prover is actually keeping the DAG on disk. These proofs are used for mining new blocks,
instead of the nonce used in the proof of work. A notion of quality is defined for
them, in such a way that the probability of deriving a proof of high quality increases
with the size of the DAG, which is an incentive to dedicating more space for mining.
While~\cite{DziembowskiFKP15} requires space to remain allocated between challenges,
and is consequently called a proof of \emph{persistent} space, \cite{AtenieseBFG14} requires
to allocate space only when answering challenges and is consequently called
a proof of \emph{transient} space (or a proof of secure erasure, as~\cite{DziembowskiFKP15} calls it).
Both solutions have an initialization phase, when the verifier performs a deeper challenge
of the prover and stores the resulting proof in blockchain, followed by an execution phase,
when the verifier challenges the prover and creates new blocks including the resulting proofs.

In the same line of~\cite{AtenieseBFG14,DziembowskiFKP15}, also~\cite{RenD16} uses
pebbling for stacked expander graphs, to get simpler, more efficient and
provably space-hard solutions.
It works for both proof of transient space and proof of persistent space.
It includes a nice review of previous proof of space
and related techniques: memory-hard functions, proof of secure erasure, provable data possession,
proof of retrievability.

In~\cite{Reyzin23}, it is studied what happens if the prover stores only a part of the file.
Namely, in that case, it should be difficult for the prover to recover the
missing portion of the file, when answering challenges from the verifier, or otherwise
the algorithm suffers from other authors call a \emph{time/memory tradeoff}.
In the context of graph pebbling, the initialization phase prevents most cheating
(that is, keeping incomplete data on disk). In~\cite{Reyzin23}, the size
of the portion of the file not kept on disk is related to the consequent time
complexity degradation for computing the missing part to answer challenges.
Ideally, the full file and pebbles must be kept in memory
for having no time complexity explosion, but they show that this is not the case for existing solutions.
They provide sufficient conditions for the initialization phase that guarantee the ideal situation.

As shown above, the use of graph pebbling seems to dominate the literature on proof of space.
However, that approach has its drawbacks.
Namely, as stated in~\cite{AbusalahACKPR17}, the proof to include in the blocks is relatively large:
kilobytes but even megabytes for the proof created in the initialization phase, for each new
prover (miner) that joins the blockchain. That initialization phase is a problem by itself: it complicates
the protocol and requires to spend cryptocurrency even \emph{before}
starting mining. Compare this with Bitcoin, for instance, that allows one to start mining
and \emph{later} collect cryptocurrency on the way. This is a barrier to the arrival of new
miners, that is, a limit to democracy.
This is why~\cite{AbusalahACKPR17} proposes an alternative technology, which is the
theoretical background of the Chia network~\cite{CohenP19,Chia}:
a proof of sequential work on top of a proof of space, based on challenges
about the inversion of a random function, for which it has been solved
the well-known issue about time-memory trade-offs. This is not, however, a pure proof of space.
Another alternative is the proof of retrievability in~\cite{JuelsK07}: a large file
is initially sent from the verifier to the prover and later the verifier
challenges, repeatedly, the prover to see if it still keeps the file in storage.
Its apparent simplicity
is jeopardized by the fact that big files must be shared among the provers (miners) and
the verifiers. In a blockchain network, they must be shared among \emph{all} (present and future)
peers, for \emph{all} (past, present and future) miners, or otherwise peers could not verify the blocks.
This is impractical, because of the large size of the files and because it complicates the protocol.

%Cite~\cite{DziembowskiFKP15}. This seems to be the first description of proofs of space.
%Their algorithm is based on graph pebbling, where a vertex can be pebbled only if the
%its in-going vertices have been pebbled as well. This way they prove a lower bound on the
%complexity of their algorithm. They prove that that lower bound is valid also if a prover
%wants to use its CPU. They prove that the size of the space used by their algorithm
%is a lower bound to the execution cost of the algorithm if no space is reserved.
%Therefore, using proofs of work in a network of proofs of space nodes would
%be computationally too expensive.
%This is sometimes called proof of persistent space.
%There is an initialization protocol for each new prover, that is missing in Burstcoin.

%Cite~\cite{AtenieseBFG14}. Based on DAGs with high pebbling complexity. There are clear similarities
%with~\cite{DziembowskiFKP15}. They actually cite and compare with each other.
%According to~\cite{DziembowskiFKP15}, this article defines a proof of secure erasure,
%that however they call a proof of space. According to~\cite{DziembowskiFKP15},
%their proof of secure erasure
%implies a proof of space but not the other way round. There does not seem to exist
%any implementation. The issue with the size of the proofs would be identical to that
%of~\cite{DziembowskiFKP15}, because of the use of pebbling graphs.
%This is sometimes called proof of transient space: the puzzle function requires lot of
%memory space to compute, but after computation that space con be freed.

An issue with proof of space is tackled in~\cite{TangZDWLG0L19}.
They observe that miners can use the same big file on disk
for mining on many chains, simultaneously. This is problematic for newborn chains, that could
get attacked from miners that already keep a very big file for mining other,
more mature chains: their history could be easily hijacked
(a so called \emph{newborn attack}). The solution in~\cite{TangZDWLG0L19} is
to split the space for mining on many chains, with an
incentive to allocate, for each chain, a space proportional
to the market value of the chain.

We are aware of only one implementation of graph-pebbling proof of space:
SpaceMint~\cite{ParkKFGAP18}, previously Spacecoin.
It must be stated that SpaceMint is not a blockchain network,
but only a prototype implementation of the proof of space protocol of~\cite{DziembowskiFKP15}.
Its code~\cite{SpaceMintCode} has not been maintained in the last nine years.
Nevertheless, \cite{ParkKFGAP18} exposes interesting problems (and solutions)
that are related to the actual game theory and implementation of proof of space.
For instance, it uses the public key of the verifier as an input parameter
of graph pebbling, to combat the creation of mining pools, often seen
negatively~\cite{MillerKKS15}.
Furthermore, it observes that proof of space, being computationally cheap, introduces
specific \emph{nothing-at-stake} problems, the most relevant being that:
%
\begin{enumerate}
\item miners might find it profitable to mine multiple chains simultaneously (not only the best chain),
  aiming at finding a better chain with a history change;
\item miners might find it profitable to mine many alternative new blocks, each holding
  different transactions, hoping to find better solutions to future challenges (\emph{block grinding}).
\end{enumerate}
%
In both cases, miners will end up mining with a mix of proof of space and proof of work,
which nullifies the benefits of the former.
In~\cite{ParkKFGAP18}, a solution to the first problem is suggested, where
previous blocks are forced to affect future blocks only in a limited way
(the next challenge is derived from the hash of a block deep in the past)
and where peers get punished if they are found to have
created more children blocks for the same father, through a penalty transaction that
includes the children blocks, for evidence.
The same problem is considered in~\cite{CohenP19}, and their solution has been
to use the same challenge for several consecutive blocks, since it is
unlikely that a challenge will be good for many consecutive blocks.
The solution that~\cite{ParkKFGAP18} suggests
for the second problem, instead, consists in making the challenges
(and consequently the proofs) independent from the content of the transactions, by splitting the
blockchain in a proofs blockchain and in a transactions blockchain: only the first is used for mining,
and the two are connected with the signature of the miner.
As a result, the next challenges (and proofs) depend on previous blocks from the proofs blockchain only.
Moreover, \cite{ParkKFGAP18} includes actual experiments, that provide an estimation of the size of the proofs.
Namely, those used in the initialization phase are between two and three megabytes, while those used
in the execution phase can be optimized to around 100 kilobytes. Such proofs must be persistently stored in
blockchain (for each new miner, in the first case, and for each new block, in the second).
This makes the size of the blockchain much larger than in the case of Bitcoin, and requires
that miners hold cryptocurrency to store proofs even before starting mining.
We observe that also the use of penalty transactions, internally including blocks, is problematic, since
transactions are, in turn, included in blocks, which makes it impossible to define
a maximal block size, a basic security requirement of every blockchain.
We have contacted the authors of~\cite{ParkKFGAP18}, reporting these issues,
but have not received any answer.

Signum~\cite{Signum}, previously Burstcoin, is another implementation of proof of space.
It is a full-fledged blockchain, launched in 2014 and still active,
not just a prototype implementation of a protocol. It includes a smart contract language, called SmartJ.
Signum is not based on graph pebbling: in its proof of space algorithm, that they
call proof of capacity, each miner allocates a large \emph{plot file} of hashes.
Peers that want to mine new blocks contact miner(s) to ask for a
\emph{deadline}\ie a compact data structure with a quality measure (its \emph{waiting time}) that,
in general, is proportional to the size of the plot files. Such deadline is
inserted in the new mined blocks as its proof of space. The simplicity of the protocol
follows from the fact that there is no initialization phase: each miner creates its plot file
independently and off-line: it is not shared and no initial big proof is stored in
blockchain, as it was the case for SpaceMint.
However, Signum has received very little scientific
attention up to now. There is not even a formalization of its proof of space algorithm,
but only an informal description~\cite{SignumPlotting} and a (largely uncommented)
implementation~\cite{SignumSource}.
In~\cite{ParkKFGAP18}, they state that Burstcoin has some
time/memory tradeoffs (``a miner doing a little extra computation can mine at the same
rate as an honest miner, while using just a small fraction (\emph{e.g.}, 10\%) of the space''),
which the developers of Signum assert to have solved. Moreover, the latest version of Signum
seems to have separated data in a proofs blockchain and a transactions blockchain,
an idea clearly borrowed from~\cite{ParkKFGAP18}. However, there is no evidence that this has been done
correctly.
%Examples can be found at \url{https://github.com/signum-network/signum-smartj}.
%While Hotmoka abstracts away all blockchain details, SmartJ requires programmers to

\section{Conclusion}\label{sec:conclusion}
%
Our formalization of Signum's consensus is valuable because it sheds light
on a blockchain network that runs since ten years but was missing any formal definition.
Moreover, it allowed us to understand that Signum is free from block grinding attacks
and is largely protected from challenge grinding attacks.

In the context of proof of space consensus, Signum's advantage is its simplicity,
the small size of its proofs (deadlines, around $200$ bytes)
and the absence of an initialization phase and transactions.
However, its main drawback is that it is
perfectly possible to mine new blocks in a proof of work style: instead of storing a plot,
Def.~\ref{def:deadline_from_plot} could recompute the nonces on the fly. This is currently
not convenient, since Alg.~\ref{alg:nonce_construction} is relatively slow, in particular
by using the shabal256 hashing algorithm for $h_\deadline$ (Tab.~\ref{tab:notations}).
But the situation might change in the future
with the use of ASICs. At the end, it will be the relative
increase of ASICs speed and memory size that will decide if Signum remains a
proof of space network or if it becomes more rational to mine it with proof of work.


%%%%%%%%%%%%%%%%%%%%%%%%%%%%%%%%%%%%%%%%%%%%
\bibliographystyle{plain}
\bibliography{biblio}

\end{document}
