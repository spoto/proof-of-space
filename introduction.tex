\section{Introduction}\label{sec:introduction}

Cite~\cite{DziembowskiFKP15}. This seems to be the first description of proofs of space.
Their algorithm is based on graph pebbling, where a vertex can be pebbled only if the
its in-going vertices have been pebbled as well. This way they prove a lower bound on the
complexity of their algorithm. They prove that that lower bound is valid also if a prover
wants to use its CPU. They prove that the size of the space used by their algorithm
is a lower bound to the execution cost of the algorithm if no space is reserved.
Therefore, using proofs of woek in a network of proofs of space nodes would
be computationally too expensive. It seems that the memory allocated in the miners
must also be allocated in the verifiers, for all miners. This would ne practically unfeasable.
To investigate further.
This is sometimes called proof of persistent space.
There is an initialization protocol for each new prover, that is missing in Burstcoin.

Cite~\cite{AtenieseBFG14}. Based on DAGs with high pebbling complexity. There are clear similarities
with~\cite{DziembowskiFKP15}. They actually cite and compare with each other.
According to~\cite{DziembowskiFKP15}, this article defines a proof of secure erasure,
that however they call a proof of space. According to~\cite{DziembowskiFKP15},
their proof os secure erasure
implies a proof of space but not the other way round. There does not seem to exist
any implementation. The issue with the size of the proofs would be identical to that
of~\cite{DziembowskiFKP15}, because of the use of pebbling graphs.
This is sometimes called proof of transient space.

Cite~\cite{TangZDWLG0L19}.

Cite~\cite{RenD16}.

Cite~\cite{ParkKFGAP18}. SpaceMint, previously Spacecoin. Consideration: PoW requires power
to be allocated if mining is worthwhile. PoS allows one to allocate unused space even if its
cost is higher than mining, since in any case it would remain unused. More egalitarian:
general-purpose hardware instead of ASIC. The use of a key for the miners makes it
impossible to build mining pools, which is said to be good, citing~\cite{MillerKKS15}.
They say PoS is more difficult to adapt to blockchain because the protocol is a bit
more complicated than PoW. It lists some problems of PoS: mining multiple chains simultaneously
(since they are cheap), creating more blocks with the same proof and then choose the most
favorable. Nothing-at-stake problems. Quality-function to determine the winner, proportional
to the allocated space. Multiple chains: penalize it. Grinding: make the proof unique, based
on who wonthe previous round, use
two chains, for proofs and for transactions, the proofs depends on previous proofs only.
Previous blocks affect future blocks only in a limited way.
They provide a game-theoretic model showing that the system is
a Nash equilibrium.
It cites proof of storage/retrievability: the verifier must send and keep a big file.
Some link with Permacoin, which is however still a PoW system with ethical data.
It cites Burstcoin (now Signum) and its time/memory tradeoff: it is still possible
to mine thorugh PoW and use only 10\% of the space.
It talks about a problem with miners hashing 8 million blocks, that does not seem to exist
anymore, but better check what they mean.
It cites the Chia Network (proof of space and time), \url{https://www.chia.net/}.
It calls it proof of sequential work on top of proof of space.
It says that it is based on completely different theoretical work, that is~\cite{AbusalahACKPR17}.
Consideration: their mining uses special protocol transactions
(payments, space commitments, penalties) while Mokamint is completey transaction agnostic.
Arrival of new miners and penalties for miners are kept in blockchain!
To avoid mining for different chains, the next challenge is derived from the hash of a block
(from the proof chain) deep in the past. If two children blocks are created by the same miner,
a penalty transaction is generated. The transaction includes the two blocks (it is huge!)
that are consequently signed, which guarantees that it can be verified by nodes that might
only have one history in the database.
The same challenge is used for a few consecutive blocks, to fight challenge-grindning attacks.
The size of their proofs (node pebbling) reaches 3 megabytes. Mokamint's deadline have constant (small size).
Code of Spacemint: \url{https://github.com/kwonalbert/spacemint}. Just a very limited prototype of a
proofs of space algorithm. Not maintained in the last nine years.

Cite~\cite{Reyzin23}. It tackles the question: how much does it cost to store only a part of the file?
When storing less than all of the file, it should be difficult for the prover to recover the
missing portions of the file when answering queries from the verifier. They define some thresholds
on the portion kept for the file and on the consequent complexity degradation. Ideally, such thresholds
should be close to 0, meaning that almost all file must be kept in memory for having no complexity
explosion. They show that existing solutions have bad constants or can be considered as impractical.
They say that the initialization protocol of the DAGs prevents most cheating (incomplete calculations).
The missing pebbles (red nodes) must be calculated later during the execution protocol.
They provide lower bounds on the cosntants of the initialization protocol so that
the resulting partial pebbling has thresholds close to 0.

Cite~\cite{DworkN92}.

Burstcoin, now Signum: \url{https://wiki.signum.network/}. Rebranded from Burstcoin.
They call it proof of capacity but it's just proof of space.
There is a very raw description of the mining algorithm:
\url{https://wiki.signum.network/signum-plotting-technical-information/index.htm}.
