\section{Introduction}\label{sec:introduction}

A blockchain is a data structure where \emph{transactions} are kept inside blocks.
Blocks form a chain (a list), where each block $b$ \emph{points} to its previous block $p$
by referring, inside $b$, to the hash of $p$. Blocks must satisfy some consistency
rules, called \emph{consensus} rules; for instance, if $b$ refers to a previous bock hash
$h$ then a block $p$ having that hash must really exist in the chain; moreover, the timestamp
of $b$ must be larger than that of $p$; the size of $b$ must be smaller than a given threshold and so on.
The exact nature of the transactions is not relevant in this paper. In general, they are
requests to update the state of a global abstract machine; this state might be a ledger of
payments (as in the case of Bitcoin~\cite{Nakamoto08,Antonopoulos17}) or a sort of global RAM where data
structures can be allocated and subsequently modified (as in the case of Ethereum~\cite{AntonopoulosW18}).

The use of hashes as machine-independent
pointers allows blockchains to be implemented in a distributed way, in a network of peers.
Distributions is a desirable property because it entails that data is safely duplicated
in each peer and that there is no special peer that determines the history of the transactions.
However, each peer is free to expand the blockchain with new blocks, independently from the other
peers, so that, in general, there
are more blocks $b$ that refer to the same previous block $p$ and the blockchain is a tree rather than
a list of blocks. In order to make a single chain emerge as the \emph{best} chain, a notion of
chain quality is used: peers have incentives to append blocks to the chain with the highest quality.
This entails that a peer might replace its current best chain with another, even better chain,
if it receives the latter from other peers. This event is a \emph{history change}.

The above description of a distributed blockchain is still missing a key ingredient. Namely,
as presented above, each peer is free to generate new blocks at maximal speed, flooding the network
with new blocks, making difficult the emergence of a best chain and inducing frequent history changes.
This is not just an efficiency problem but also a security problem: history changes allow
\emph{double spending}, when the same money is moved in the ledger twice, once in the previous history
and once in the updated history. The actual genious of Nakamoto~\cite{Nakamoto08} has been to (largely)
solve these issues in a very elegant way, by exploiting an idea previously developed for combatting
email spam~\cite{DworkN92}. Namely, he added a consensus rule requiring that the (binary)
hash of each block $b$ must start with at least $\delta$ zeros, and connected the quality of a chain
to this hash. This means that the creation of a new block requires to rotate among
many possible values for a block field, called \emph{nonce}, until the hash of the block satisfies the
added consensus rule. This makes the creation of new blocks hard
(for larger $\delta$), makes it impossible to create blocks at arbitrary speed and creates a heavy
incentive to expanding the best chain, rather than creating alternative histories, since otherwise a peer
risks spending work (concretely, electricity) for the creation of blocks that will be discarded by the other peers.
The value $\delta$ is called \emph{difficulty} and is not a constant: it changes in accordance with
the current, total computational power of the network, in order to keep the block creation rate at a predetermined value.
The process of finding a good nonce, that induces an
acceptable block hash, is the \emph{proof of work} algorithm: it is a brute force algorithm, because of the non-correlation
property of hash functions. A peer that performs the proof of work algorithm is said to \emph{mine}
a new block and is consequently a block \emph{miner}. Miners get remunerated for their work whenever
they mine a new block before all other miners. Theoretically, everybody can install a miner peer,
anonymously, which makes the idea of Bitcoin very democratic.

The proof of work secures a blockchain network, reducing the risk of double spending,
but comes at the price of energy consumption: the electricity used by the Bitcoin network is said
to be comparable to that of a medium-sized country; moreover, mining is not egalitarian, because
it is worthwhile only in countries where electricity is cheap; furthermore, the proof of work algorithm
is more efficient in dedicated, relatively expensive hardware (such as ASICs),
which deviates much from the idea of a democratic and open network.
