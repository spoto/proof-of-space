\section{Introduction}\label{sec:introduction}

Cite~\cite{DziembowskiFKP15}. This seems to be the first description of proofs of space.
Their algorithm is based on graph pebbling, where a vertex can be pebbled only if the
its in-going vertices have been pebbled as well. This way they prove a lower bound on the
complexity of their algorithm. They prove that that lower bound is valid also if a prover
wants to use its CPU. They prove that the size of the space used by their algorithm
is a lower bound to the execution cost of the algorithm if no space is reserved.
Therefore, using proofs of woek in a network of proofs of space nodes would
be computationally too expensive. It seems that the memory allocated in the miners
must also be allocated in the verifiers, for all miners. This would ne practically unfeasable.
To investigate further.

Cite~\cite{AtenieseBFG14}.

Cite~\cite{TangZDWLG0L19}.

Cite~\cite{RenD16}.

Cite~\cite{ParkKFGAP18}. Previously Spacecoin, it seems.

Cite~\cite{Reyzin23}.

Cite~\cite{DworkN92}.

Code of Spacemint: \url{https://github.com/kwonalbert/spacemint}. Just a very limited prototype of a
proofs of space algorithm. Not maintained in the last nine years.
