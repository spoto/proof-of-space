\section{Blockchain Construction}\label{sec:blockchain_construction}

The blocks of the blockchain contain information used for consensus,
called \emph{trunk} by borrowing this terminology from~\cite{CohenP19};
other information such as the previous block hash;
and extra information that is irrelevant here, such as a list of transactions,
that are not formalized below since they are not used by Signum's consensus.
Blocks can be genesis and non-genesis. Both contain their time of creation and their acceleration.
Genesis blocks have no trunk nor parent; their height is implicitly $0$.
Challenges $c$ are generated in sequence: there is an initial constant challenge for the genesis
block, while the challenge of non-genesis blocks $b$ is generated from their trunk.
In particular, $c$ is \emph{not} computed from the transactions in $b$,
in order to avoid block-grinding attacks (Sec.~\ref{sec:prolog_structure_attacks}).
A deadline that answers $c$ is recorded in the trunk of the sons of $b$.
%
\begin{definition}[Trunk, Block]\label{def:trunk}
  The sets of \emph{trunks} and \emph{blocks} are
  \begin{align*}
    \Trunks&=\left\{\langle\height,\delta\rangle\left|\;\height\in\mathbb{N}\text{ and }\delta\in\Deadlines\right.\right\},\\
    \GenesisBlocks&=\left\{\langle\tau,\alpha\rangle\mid\tau\in\mathbb{N},\ \alpha\in\mathbb{N}\text{ and }\alpha>0\right\},\\
    \NonGenesisBlocks&=\left\{\left\langle\begin{array}{c}
    \tau,\alpha,\power,\\
    \weightedbeat,\trunk,\\
    \previousblockhash
    \end{array}\right\rangle\left|\begin{array}{l}
    \tau\in\mathbb{N},\\
    \alpha\in\mathbb{N},\ \alpha>0,\\
    \power\in\mathbb{N},\\
    \weightedbeat\in\mathbb{N},\\
    \trunk\in\Trunks,\\
    \previousblockhash\text{ is a}\\
    \qquad\text{hash for $h_\block$}
    \end{array}\right.\right\},\\
    \Blocks&=\GenesisBlocks\cup\NonGenesisBlocks.
  \end{align*}
\end{definition}
%
If $b$ is a block, then $b.\tau$ is its creation time (milliseconds from the Unix epoch)
and $b.\alpha$ is the acceleration at $b$. If $b$ is a non-genesis block, then
$b.\power$ expresses how much space has been used to build the path that leads to $b$,
starting from the genesis block; it will be used to select the \emph{best chain} for mining
$b$'s sons. The value of $b.\weightedbeat$ is the average of the interval creation time
for the path that leads to $b$; it weighs the last blocks more. It will be
compared to $\beat$ (Tab.~\ref{tab:notations}) to understand if the acceleration
must be increased or decreased in $b$'s sons.
The value of $b.\previousblockhash$ is the hash of the previous block in the path leading to $b$.
If $b$ is a genesis block, we abuse notation and assume that $b.\power=b.\weightedbeat=0$.

\begin{definition}[Block's height]\label{def:block_height}
  Let $b\in\Blocks$. The \emph{height} of $b$ is defined as
  \[
  \height(b)=\begin{cases}
  0 & \text{if $b\in\GenesisBlocks$}\\
  b.\trunk.\height & \text{if $b\in\NonGenesisBlocks$.}
  \end{cases}
  \]
\end{definition}
%
Def.~\ref{def:initial_challenge} shows how the first challenge is defined, for genesis blocks.
It is a constant that only depends on contextual values (Table~\ref{tab:notations}).
%
\begin{definition}[$\initialchallenge$]\label{def:initial_challenge}
  The \emph{initial challenge} is the constant\footnote{In~\cite{SignumPlotting},
  the scoop number is derived from $\sigma_\genesis$; we simplify it to $0$, without loss
  of generality.}
  %
  \[
    \initialchallenge=
    \langle 0,\sigma_\genesis\rangle
    %\\\langle\betonat(h_\generation(\sigma_\genesis\append\nattobe(1)))\text{ mod }\numberofscoops,\sigma_\genesis\rangle,
  \]
  %
  where $\sigma_\genesis$ is a constant generation signature used for the genesis of the blockchain
  (see Table~\ref{tab:notations}).
\end{definition}
%
Def.~\ref{def:next_challenge_from_trunk}
shows how a challenge is derived from the trunk of a non-genesis block.
%
\begin{definition}[$\nextchallenge(\trunk)$]\label{def:next_challenge_from_trunk}
  Let $\trunk\in\Trunks$. The \emph{next challenge for $\trunk$} is
  \begin{multline*}
    \nextchallenge(\trunk)=\\
    \langle\betonat(h_\generation(\sigma\append\nattobe(\trunk.\height+1)))\text{ mod }\numberofscoops,\sigma\rangle
  \end{multline*}
  where
  \[
  \sigma=h_\generation(\trunk.\delta.\challenge.\sigma\append\trunk.\delta.\pi).  
  \]
\end{definition}
%
The construction of the generation signature $\sigma$ for the next challenge,
in Def.~\ref{def:next_challenge_from_trunk},
has puzzled us for some time, since~\cite{SignumPlotting} appends a \emph{previous block generator}
to the previous block's generation signature $\trunk.\delta.\challenge.\sigma$.
That concept, however, is defined nowhere.
We had to dive in the source code of Signum to understand that it is actually
an identifier (more concretely, the public key)
of the creator of the deadline for the previous block
(see \url{https://github.com/signum-network/signum-node/blob/main/src/brs/GeneratorImpl.java},
in the constructor of \<GeneratorStateImpl>).
Our prolog of the previous deadline generalizes that information, hence
Def.~\ref{def:next_challenge_from_trunk} appends $\trunk.\delta.\pi$ to define $\sigma$.

Later, it will be handy to determine the next challenge for a block. Note that
it only uses the trunk inside the block.
%
\begin{definition}\label{def:next_challenge_from_block}
  Let $b\in\Blocks$. Its \emph{next challenge} is
  \[
  \nextchallenge(b)=\begin{cases}
  \initialchallenge & \text{if $b\in\GenesisBlocks$}\\
  \nextchallenge(b.\trunk) & \text{if $b\in\NonGenesisBlocks$.}
  \end{cases}
  \]
\end{definition}

Def.~\ref{def:next} shows how the information inside a block is used to construct
that inside its sons. It is actually our proposal, since there is no information
in~\cite{SignumPlotting} about this. The computation of the next weighted beat
gives more or less weight to the previous weighted beat, depending on a constant $\oblivion$
($0\le\oblivion\le 1$) that expresses how quickly the acceleration reacts to changes
in mining power. The computation of the next power uses the same formula
as Bitcoin~\cite{WalkerG24}, adapted to our context: the ratio between the maximal (hence worse)
deadline's value $2^{8\cdot\size(h_\deadline)}$ and the actual deadline's value
expresses how much space has been used to compute the deadline.
%
\begin{definition}[Next functions]\label{def:next}
  Let $b\in\Blocks$ and $\delta\in\Deadlines$. We define
  %
  \begin{align*}
    \nextwaitingtime(b,\delta)&=b.\tau+\waitingtime(\delta,b.\alpha),\\
    \nextweightedbeat(b,\delta)&=\begin{array}{c}
    \waitingtime(\delta,b.\alpha)\cdot\oblivion\\
    +b.\weightedbeat\cdot(1-\oblivion)
    \end{array},\\
    \nextacceleration(b,\delta)&=\frac{b.\alpha\cdot\nextweightedbeat(b,\delta)}{\beat},\\
    \nextpower(b,\delta)&=b.\power+\frac{2^{8\cdot\size(h_\deadline)}}{\betonat(\delta.\mathit{value})+1}.
  \end{align*}
\end{definition}
%
Def.~\ref{def:next_block} shows how a next block is constructed, once
its deadline has been chosen.
%
\begin{definition}[Next block]\label{def:next_block}
  Let $b\in\Blocks$ and $\delta\in\Deadlines$. We define
  $\nextblock(b,\delta)\in\NonGenesisBlocks$ as
  \[
  \nextblock(b,\delta)=\left\langle\begin{array}{c}
  \nextwaitingtime(b,\delta),\nextacceleration(b,\delta),\\
  \nextpower(b,\delta),\nextweightedbeat(b,\delta),\\
  \langle\height(b)+1,\delta\rangle,h_\block(b)
  \end{array}\right\rangle,
  \]
  where $h_\block(b)$ is the application of $h_\block$ to the byte representation of $b$.
\end{definition}

A blockchain is a set of blocks, linked through their $\previousblockhash$ field.
It must contain exactly one genesis block; there is no hash collision among its blocks;
and all its blocks must satisfy the \emph{consensus rules}.
%
\begin{definition}[Blockchain, Consensus]\label{def:blockchain}
  A \emph{blockchain} is a set $B\subseteq\Blocks$ such that:
  \begin{enumerate}
  \item\label{prop:blockchain:genesis} there is exactly one $b\in B\cap\GenesisBlocks$, written as $\genesis(B)$;
  \item\label{prop:blockchain:no_collision} for each hash $h$ of $h_\block$, there is at most
    one $b\in B$ such that $h_\block(b)=h$, written as $\block(B,h)$;
  \item\label{prop:blockchain:consensus} for each $b\in B$, the predicate $\consensus(B,b)$ holds, where
    %
    \begin{itemize}
    \item if $b\in\GenesisBlocks$, then $\consensus(B,b)$ is only the consensus rule:
      \begin{enumerate}[label=(\alph*)]
      \item $b$ is not created in the future:
        \[
        b.\tau\le\tau_\now;
        \]
      \end{enumerate}
    \item if $b\in\NonGenesisBlocks$, then $\consensus(B,b)$ is the logical conjunction
      of all the following consensus rules:
      \begin{enumerate}[label=(\alph*)]
      \item\label{prop:consensus:no_future} $b$ is not created in the future:
        \[
        b.\tau\le\tau_\now;
        \]
      \item\label{prop:consensus:valid} the deadline of $b$ (that is, $b.\trunk.\delta$) is valid (Def.~\ref{def:deadline_validity});
      \item\label{prop:consensus:no_dangling} there are no dangling pointers:
        \[
        p=\block(B,b.\previousblockhash)\text{ exists;}
        \]
      \item\label{prop:consensus:answer} the deadline of $b$ answers the challenge of $p$ (Def.~\ref{def:next_challenge_from_block}):
        \[
        \nextchallenge(p)=b.\trunk.\delta.\challenge;
        \]
      \item\label{prop:consensus:next_block} $b$ is the next block of $p$ \wrt the deadline of $b$ (Def.~\ref{def:next_block}):
        \[
        b=\nextblock(p,b.\trunk.\delta).
        \]
      \end{enumerate}
    \end{itemize}
    %
  \end{enumerate}
\end{definition}
%
The above consensus rules, reconstructed and interpolated
from~\cite{SignumPlotting,SignumSource},
do not constrain the prologs of the deadlines:
each block can have an arbitrary prolog. Sec.~\ref{sec:prolog_structure_attacks}
will show why it is useful to restrain prologs with extra consensus rules.
%
\begin{definition}[Blockchain network]\label{def:blockchain_network}
  A \emph{blockchain network} is a
  network of \emph{peers} (computers), each connected to the other peers,
  each holding its own vision of the blockchain, all for the same genesis block.
  Each peer holds a plot (Def.~\ref{def:plot}) in its memory, starts with
  a blockchain that only holds the genesis block and runs,
  concurrently, the \emph{block mining} algorithm
  and the \emph{block mined} algorithm.
\end{definition}
%
Def.~\ref{def:blockchain_network} simplifies the picture very much:
peers are fully connected, never disconnect and never need to synchronize.
Moreover, in practice, peers do not hold plots but rather rely on (one or more)
external services (miners) that hold one or more plots.
The goal here is to keep the picture as simple as possible and concentrate on the properties of
Signum's consensus only: Def.~\ref{def:blockchain_network} does not pretend
to describe a real blockchain implementation in detail.

The block mining algorithm, referred in Def.~\ref{def:blockchain_network},
looks for the most powerful block in blockchain,
mines a new block on top of it, adds it to the blockchain and whispers it to the connected peers.
%
\begin{alg}[Block mining]\label{alg:block_mining}
  The \emph{block mining} algorithm of a peer $P$, holding blockchain $B$,
  is the following infinite loop:
  %
  \begin{enumerate}
  \item\label{step:block_mining:loop} identify a most powerful\footnote{In theory, more \emph{most powerful blocks} might exist in blockchain, although this is highly unlikely; in that case, step~\ref{step:block_mining:loop} will choose any of them.} block $p$ in $B$;
  \item\label{step:block_mining:next_challenge} compute $c=\nextchallenge(p)$ (Def.~\ref{def:next_challenge_from_block});
  \item\label{step:block_mining:next_deadline} compute $\delta'=\delta(\plot,c)$ (Def.~\ref{def:deadline_from_plot}), where $\plot$
    if the plot of $P$;
  \item\label{step:block_mining:next_block} compute $b'=\nextblock(p,\delta')$ (Def.~\ref{def:next_block});
  \item\label{step:block_mining:wait} wait until $b'.\tau\le\tau_\now$;
  \item\label{step:block_mining:add} add $b'$ to $B$;
  \item whisper $b'$ to the peers connected to $P$;
  \item go back to step~\ref{step:block_mining:loop}.
  \end{enumerate}
\end{alg}
%
The block mined algorithm receives a block whispered from a connected peer, checks its validity
and adds it to the blockchain.
%
\begin{alg}[Block mined]\label{alg:block_mined}
  The \emph{block mined} algorithm of a peer $P$, holding blockchain $B$,
  is the following infinite loop:
  %
  \begin{enumerate}
  \item\label{step:block_mined:loop} wait for a block $b$ whispered from a connected peer $P'$;
  \item\label{step:block_mined:add} if $B\cup\{b\}$ is a blockchain, add $b$ to $B$;
  \item go back to step~\ref{step:block_mined:loop}.
  \end{enumerate}
\end{alg}
%
In practice, step~\ref{step:block_mined:add} of Alg.~\ref{alg:block_mined} could
only allow the addition of $b$ if it looks \emph{powerful enough},
in order to avoid filling the memory with useless blocks. This is not relevant here.
Moreover, if the whispered block $b$
at step~\ref{step:block_mined:add} of Alg.~\ref{alg:block_mined} is more
powerful than $b'$ at step~\ref{step:block_mining:wait} of Alg.~\ref{alg:block_mining},
a rational peer would interrupt the wait at step~\ref{step:block_mining:wait}
of Alg.~\ref{alg:block_mining}, discard $b'$ and restart Alg.~\ref{alg:block_mining}
from step~1, since the whispered $b$ is better than the block $b'$ that it is trying to mine,
hence it is wiser to stop waiting and start mining on top of the new best block $b$.
Such optimizations are not considered here.

Peers check the validity of blocks coming from outside
(step~\ref{step:block_mined:add} of Alg.~\ref{alg:block_mined}), since they do not trust
their connected peers. Instead, they do not check the validity of the blocks that they
mine themselves (step~\ref{step:block_mining:add} of Alg.~\ref{alg:block_mining}),
since they are valid by construction. Namely,
Prop.~\ref{prop:mining_is_sound} guarantees that $B$ remains a blockchain in every peer.
The hypothesis on no hash collision is standard for blockchains.
%
\begin{proposition}\label{prop:mining_is_sound}
  If no hash collision occurs at step~\ref{step:block_mining:add} of Alg.~\ref{alg:block_mining},
  then the set $B$ of blocks in each peer is a blockchain.
\end{proposition}
