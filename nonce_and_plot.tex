\section{Nonces and Plots}\label{sec:nonces_and_plots}

This section formalizes the notion of nonce and plot, together with their
algorithmic construction. Tab.~\ref{tab:notations} collects some notations,
introduced here and used throughout the paper.
It also reports the specific choices
made in~\cite{SignumPlotting} for such notations, but this paper remains parametric
\wrt them.

\begin{table}[t]
\begin{center}
\begin{tabular}{|c|c|c|}
  \hline
  \textbf{notation} & \textbf{meaning} & \textbf{in~\cite{SignumPlotting}}\\\hline\hline

  $\append$ & concatenation of sequences of bytes & \\\hline
  $\numberofscoops$ & number of scoops contained in a nonce & $4096$\\\hline

  $h_\deadline$ & hashing function for computing nonces, plots and deadlines & shabal256\\\hline

  $h_\generation$ & hashing function for computing the generations of challenges & shabal256\\\hline

  $\kappa$ & maximal number of bytes fed to $h_\deadline$ & $4096$\\\hline

\end{tabular}
\end{center}
\caption{Notations used in our formalization and their specific instantiations
  used in~\cite{SignumPlotting}.}
\label{tab:notations}
\end{table}
%
\begin{definition}[bytes concatenation $\bowtie$]
  Sequences of bytes are concatenated by $\bowtie$. We abuse notation
  and apply $\bowtie$ to natural numbers as well, meaning that
  the big-endian byte representation of the numbers gets concatenated.
\end{definition}
%
The definition of nonces and plots is parametric \wrt a hashing
function $h_\deadline$ used for their creation.
%
\begin{definition}[hashing function]
  A \emph{hashing function} $h$ of $\size>0$
  is a total map $h:\mathit{byte}^*\to\mathit{byte}^\size$, where
  $\mathit{byte}^\size$ is a sequence of $\size$ bytes, called a \emph{hash} for $h$.
  If $h$ is a hashing function, then $\size(h)$ is its size.
\end{definition}
%
\begin{definition}[scoop, nonce]
  A \emph{scoop} is a pair of hashes for $h_\deadline$. A \emph{nonce} $n$ is a list of
  $\numberofscoops>0$ scoops or, equivalenty, a list of $2\times\numberofscoops$
  hashes $n_1,\ldots,n_{2\times\numberofscoops-1}$ for $h_\deadline$ or, equivalently, a sequence of
  $2\times\numberofscoops\times\size(h_\deadline)$ bytes.
\end{definition}
%
The construction of a nonce is parametric \wrt the non-negative progressive number
$n$ of the nonce and a sequence $\pi$ of bytes, called \emph{prolog}.
That construction if defined through an algorithm, that
uses a constant $\kappa>0$ to limit its computational cost, to avoid hashing
very large chunks of data.
%
\begin{definition}[$\nonce(n,\pi)$]
  Let $n\in\mathbb{N}$ and $\pi\in\mathit{byte}^*$. 
  The $n$th nonce with prolog $\pi$, written
  $\nonce(n,\pi)$, is constructed as follows.
  %
  \begin{enumerate}
  \item Let $\seed=\pi\append n$.
  \item For each $i$ from $2\times\numberofscoops-1$ to $0$, let
    $x$ be the first\footnote{In~\cite{SignumPlotting}, it is said to
  take the \emph{last} $\kappa$ bytes, but an inspection of their code
  shows that they actually take the \emph{first} $\kappa$ bytes. They
  probably use \emph{last} here in the sense of \emph{more recently computed}.}
    $\kappa$ bytes of
    \[
    \left(\append_{j=i+1}^{2\times\numberofscoops-1}\nonce(n,\pi)_j\right)\append\seed
    \]
    and assign $\nonce(n,\pi)_i=h(x)$.
  \item Define
    \[
    \finalhash=h\left(\left(\append_{j=0}^{2\times\numberofscoops-1}\nonce(n,\pi)_j\right)\append\seed\right).
    \]
  \item For each $i$ from $0$ to $2\times\numberofscoops-1$, reassign
    $\nonce(n,\pi)_i$ to $\nonce(n,\pi)_i\xor\finalhash$.
  \item For each odd $i$ from $1$ to $2\times\numberofscoops-1$, swap
    $\nonce(n,\pi)_i$ with $\nonce(n,\pi)_{2\times\numberofscoops-i}$.
  \end{enumerate}
\end{definition}
%
A \emph{plot} is a set of nonces for a finite subset of progressive numbers,
all for the same prolog. Both the prolog and the progressive of each nonce
is recorded in the plot.
%
\begin{definition}[$\plot(S,\pi)$]
  Let $S\subset\mathbb{N}$ be finite and $\pi\in\mathit{byte}^*$.
  The \emph{plot} for $S$ and $\pi$, written $\plot(S,\pi)$, is
  \[
  \plot(S,\pi)=\left<\pi,\left\{\left<n,\nonce(n,\pi)\right>\left|\;n\in S\right.\right\}\right>.
  \]
\end{definition}
