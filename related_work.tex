\section{Related Work}\label{sec:related_work}

Proof of work was originally meant as protection against email spam~\cite{DworkN92}:
email senders must perform some work to have emails accepted by their recipient.
%The input includes
%the address of the recipient and the date of sending, in order to ban
%recycling of work.
%The algorithm adds extra data at the end of the email,
%which corresponds to the nonce used in Bitcoin.
Ethereum started with proof of work~\cite{AntonopoulosW18} and later
moved to proof of stake. The latter can be seen as a
Byzantine consensus algorithm, as pioneered by Tendermint~\cite{Kwon14}.
Most current blockchains use some form of proof of stake nowadays.

The theoretical background of proof of space was independently developed
in two seminal papers~\cite{AtenieseBFG14,DziembowskiFKP15}.
They feature similarities but also significant differences. Both are based
on directed acyclic graphs (DAGs) of high pebbling complexity.
Pebbling, here, is a directed hash decoration of the nodes of the DAG, as in a Merkle tree.
A prover must keep such a (big) DAG and its pebbling on disk, in order to answer, efficiently,
challenges with proofs that should convince a verifier that
the prover is actually keeping the DAG on disk.
While~\cite{DziembowskiFKP15} requires space to remain allocated between challenges
(proof of \emph{persistent} space), \cite{AtenieseBFG14}~requires one
to allocate space only when answering challenges
(proof of \emph{transient} space or a \emph{proof of secure erasure}, as~\cite{DziembowskiFKP15} calls it).
Both solutions have an initialization phase, when the verifier performs a deeper challenge
of the prover and stores the resulting (big) proof in blockchain, followed by an execution phase,
when the verifier challenges the prover for each new block.
Also~\cite{RenD16} uses
pebbling for stacked expander graphs, to get simpler, more efficient and
provably space-hard solutions.
It works for both proof of transient space and proof of persistent space.
It includes a nice review of previous proof of space
and related techniques: memory-hard functions, proof of secure erasure, provable data possession,
proof of retrievability.

Time/memory tradeoffs are studied in~\cite{Reyzin23}. They occur if
the prover stores only a part of the file on disk.
A good proof of space algorithm should make it difficult for a prover to recover the
missing part, when answering challenges.
In the context of graph pebbling, the initialization phase prevents most cheating
(that is, keeping incomplete data on disk). In~\cite{Reyzin23}, the size
of the portion of the file not kept on disk is related to the consequent time
complexity degradation for computing the missing part.
Ideally, the full file and pebbles must be kept in memory
for having no time complexity explosion, but they show that this is not the case in existing solutions.
They provide sufficient conditions for the initialization phase, that guarantee the ideal result.

The use of graph pebbling seems to dominate the literature on proof of space.
Therefore, \cite{AbusalahACKPR17}~proposes an alternative technology, which is the
theoretical background of the Chia network~\cite{CohenP19,Chia}:
a proof of sequential work on top of a proof of space, based on challenges
about the inversion of a random function, for which time/memory tradeoffs have been solved.
This is not, however, a pure proof of space.
Another alternative is the proof of retrievability in~\cite{JuelsK07}: the verifier sends, initially,
a large file to the prover (miner) and later challenges, repeatedly, the prover to see
if it still keeps the file in storage. Its apparent simplicity
is jeopardized by the fact that big files must be shared.
In a blockchain network, they must be shared among \emph{all} (present and future)
peers, for \emph{all} (past, present and future) miners, or otherwise peers could not verify the blocks.

We are aware of only one implementation of graph-pebbling proof of space:
SpaceMint~\cite{ParkKFGAP18}, previously Spacecoin~\cite{ParkPAFG15}.
It is not a blockchain implementation,
but a prototype of the proof of space protocol of~\cite{DziembowskiFKP15}.
Its code~\cite{SpaceMintCode} has not been maintained in the last nine years.
Nevertheless, \cite{ParkKFGAP18}~exposes interesting problems (and solutions)
that are related to the actual game theory and implementation of proof of space.
For instance, it uses the public key of the verifier as an input parameter
for graph pebbling, to combat the creation of mining pools, often seen
negatively~\cite{MillerKKS15}.
Furthermore, it studies nothing-at-stake problems, that increase the risk of double spending
and push miners to mine with a mix of proof of space and proof of work,
thus nullifying the benefits of the former, and solutions to such problems:
%
\begin{enumerate}
\item Miners might find it profitable to mine multiple chains simultaneously (not only the best chain);
  their solution is spot such behavior and impose penalty transactions against the culprit.
  Such transactions include paits of blocks, as evidence, which is
  problematic since transactions are, in turn, included in blocks and this makes it impossible to define
  a maximal block size, a basic security requirement of every blockchain.
  We have discussed this with K.\ Pietrzak who suggested that this could be actually optimized
  by reporting signed hashes only.
\item Miners might find it profitable to mine many alternative new blocks, each holding
  different transactions, hoping to find better answers to future challenges (block grinding): their
  solution is to make the challenge independent from the transactions in the block,
  by splitting the
  blockchain in a proofs blockchain and in a transactions blockchain: only the first is used for mining,
  and the two are connected with the signature of the miner.
\item Miners might find it profitable to provide suboptimal answers to the current challenge if
  this allows them to provide much better answers later (challenge grinding): their solution is
  to let past blocks influence the quality of short sequences of future blocks only.
  The solution in~\cite{CohenP19} is more drastic: use the same challenge for several consecutive blocks,
  since it is unlikely that a challenge will be good for many consecutive blocks.
\end{enumerate}
%
Actual experiments are reported in~\cite{ParkKFGAP18},
that provide an estimation of the size of Spacemint's proofs:
in the initialization phase, they are between two and three megabytes;
in the execution phase, they can be optimized to around 100 kilobytes.
Such proofs must be persistently stored in
blockchain (for each new miner, in the first case, and for each new block, in the second).
This makes the size of the blockchain much larger than in Bitcoin, and requires
that miners hold cryptocurrency to store proofs even before starting mining.

Signum~\cite{Signum}, previously Burstcoin, is
a full-fledged blockchain based on proof of space, including a smart contract language,
launched in 2014 and still active.
A formalization of an old version of Alg.~\ref{alg:nonce_construction}
is reported in Appendix B of~\cite{ParkKFGAP18}.
The same paper states that Burstcoin has some
time/memory tradeoffs (``a miner doing a little extra computation can mine at the same
rate as an honest miner, while using just a small fraction (\emph{e.g.}, 10\%) of the space''),
which the developers of Signum assert to have solved. Moreover, the latest version of Signum
seems to have separated data in a proofs blockchain and a transactions blockchain,
an idea clearly borrowed from~\cite{ParkKFGAP18}. However, there is no evidence that this
has been done correctly.

Newborn attacks are considered in~\cite{TangZDWLG0L19}. Their solution is
to split the space for mining on many chains, with an
incentive to allocate, for each chain, a space proportional
to the market value of the chain.
