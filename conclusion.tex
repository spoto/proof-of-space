\section{Conclusion}\label{sec:conclusion}
%
Our formalization of Signum's consensus is valuable because it sheds light
on a blockchain network that runs since ten years but was missing any formal definition.
Moreover, it allowed us to understand that Signum is free from block-grinding attacks
and is largely protected from challenge-grinding attacks.

In the context of proof of space consensus, Signum's advantage is its simplicity,
the small size of its proofs (deadlines, around $200$ bytes)
and the absence of an initialization phase and transactions.
However, its main drawback is that it is
perfectly possible to mine new blocks in a proof of work style: instead of storing a plot,
Def.~\ref{def:deadline_from_plot} could recompute the nonces on the fly. This is currently
not convenient, since Alg.~\ref{alg:nonce_construction} is relatively slow, in particular
by using the notoriously slow shabal256 hashing for $h_\deadline$ (Tab.~\ref{tab:notations}).
But the situation might change in the future
with the use of ASICs. At the end, it will be the relative
increase of ASICs speed and memory size that will decide if Signum remains a
proof of space network or if it becomes more rational to mine it with proof of work.
