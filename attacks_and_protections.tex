\section{Attacks and Protections}\label{sec:attacks_and_protections}
%
In~\cite{ParkKFGAP18}, one ciriticism to Burstcoin was that
the validation of a deadline
(Def.~\ref{def:deadline_validity}) requires to run
Alg.~\ref{alg:nonce_construction}, that in turn
requires hashing $8\cdot 10^6\cdot 32=256000000$ bytes.
It must be stated that
such hashing can actually be computed in milliseconds nowadays,
with minimal energy cost. Therefore, we do not see this as a problem.
In any case, step~\ref{step:nonce_construction:first_hash} of
Alg.~\ref{alg:nonce_construction} currently uses a threshold $\kappa$,
that was possibly missing at the time of writing~\cite{ParkKFGAP18}.
Considering that threshold and assuming the specific values and hashing
used in~\cite{SignumPlotting} (see rightmost column of Tab.~\ref{tab:notations}),
step~\ref{step:nonce_construction:first_hash}
hashes at most $\kappa$ bytes and is iterated $2\cdot\numberofscoops$ times, that is,
it hashes at most $33554432$ bytes.
Step~\ref{step:nonce_construction:final_hash} hashes $32\cdot 2\cdot 4096$ bytes
plus the size of the $\seed$, which is around $200$ bytes.
That is, it hashes $262344$ bytes. In total, Alg.~\ref{alg:nonce_construction}
hashes $33816776$ bytes, which is around $8$ times less than what reported
in~\cite{ParkKFGAP18}.

This gives us the opportunity to discuss the observation in~\cite{ParkPAFG15}, page~28,
where it is stated that ``this previous block generator seems possible to be grinded, by trying
different sets of transactions to include in a block''. This is false, although it was
hard to grasp without the formalization in
Def.~\ref{def:next_challenge_from_trunk}, where it is clear that
$\trunk.\delta.\pi$ comes from the trunk only, which makes
such grinding attack impossible.
